\section{Introduction}

Reinforcement learning represents one of the most conceptually challenging areas within machine learning education. While the mathematical foundations of Markov Decision Processes and temporal difference learning are well-established, students often struggle to develop intuitive understanding when confronted solely with equations and algorithmic descriptions. The abstract nature of concepts such as value functions, policy optimization, and the exploration-exploitation trade-off becomes particularly difficult to grasp without direct observation of the learning process in action.

Traditional reinforcement learning education compounds these challenges through reliance on complex programming environments and non-real-time training paradigms. Students typically encounter reinforcement learning through frameworks such as Gymnasium\cite{gymnasium2023}, where implementation requires substantial Python programming expertise and familiarity with machine learning libraries. Training processes often demand GPU acceleration and extended computation times, leaving students to interpret learning progress through post-hoc analysis of performance graphs and numerical metrics rather than observing the dynamic decision-making process as it unfolds.

Furthermore, production-level reinforcement learning implementations expose learners to overwhelming parameter spaces, with dozens of hyperparameters and architectural choices that obscure fundamental learning principles. This complexity creates significant cognitive overhead, preventing students from focusing on core concepts such as how learning rate affects the speed of learning, how discount factor influences long-term planning, and how exploration strategies impact discovery of optimal behaviors.

This work addresses these pedagogical limitations through the development of a browser-based, real-time reinforcement learning environment that prioritizes conceptual understanding over implementation complexity. By providing immediate visual feedback, interactive parameter control, and a simplified interface that exposes only essential learning components—learning rate, discount factor, exploration rate, and reward structure—the system enables students to observe and understand reinforcement learning behavior without the barriers typically associated with traditional educational approaches.